%% ------------------------------------------------------------------------- %%
\chapter{Da Introdução}
\label{cap:introducao}

Com o advento de tecnologias  ligadas a indústria de games, é possível perceber
uma constante preocupação com a imersão do jogador na história. Isso é ótimo, e
o próprio autor se beneficia com experiências mais realistas. Mas fica uma
pergunta: será que jogos de mundo aberto, muitas vezes capazes de transpor as
barreiras temporais e espaciais poderão substituir os livros? Pessoalmente,
espero que não aconteça, mas é inegável o apelo desse jogos: a possiblidade do
jogador construir sua própria narrativa.

Os livros não oferecem essa possibilidade, mas oferecem um bem ainda maior: o
hábito da leitura que possibilita um ciclo virtuoso que envolve cultura e
conhecimento e mais leitura. Seriam a diversão e a leitura inconsiliáveis? Essa
proposta de trabalho busca provar que não. O software a ser desenvolvido permite
que se criem aventuras de texto, também chamadas de ficção interativa que será
explicada em mais detalhes no capítulo sobre os conceitos principais.

Essa solução tem dois objetivos: criar uma ferramenta de geração de aventuras de
texto de uma forma que, mesmo pessoas não familiarizadas com programação possam
criar aventuras da forma como planejarem. Outro objetivo importantíssimo a
considerar é criar um atrativo a mais para o hábito da leitura. Como cientista
da computação, não posso me atribuir a autoridade de dizer o que é ou não
benéfico para crianças. Entretanto, acredito que trazer a interatividade dos
jogos para leitura pode ter um enorme potencial para incentivar o hábito da
leitura.

\section{Objetivos Específicos}
\label{sec:goals}

Aqui tratarei de todos os objetivos de forma específica e como pretendo
cumpri-los.

\textbf{Gerar divertimento para o público} desse software. Este público não é
homogêneo, e engloba pessoas criativas, mas não exatamente familiarizadas com a
tarefa de gerar código. Além desse grupo, há também aqueles que irão se divertir
com as aventuras criadas.

Para o desenvolvedor, o atrativo será criar jogos de uma forma que deverá ser a
mais intuitiva possível. Para tal, inspirei-me no estilo de programar do
\emph{RSpec}, e também na forma de descrever os elementos do \emph{ZIL}. A ideia
é trabalhar com blocos descritivos com tokens de começo e fim (que torne esses
elementos o mais explícitos possível) e no meio destes somente pares de chaves e
seus respectivos valores.

Para o jogador, é crucial que ele tenha a melhor experiência de jogo possível.
Clássicos como a série de aventuras de texto \emph{Zork} \citep{Zork} são um bom
exemplo. Uma ideia lançada pelo orientador permitira que se inserissem sons e
planos de fundo ou imagens. É uma boa adição ao sistema, mas não é a prioridade.

\textbf{Compartilhar conhecimento.} É fundamental que tudo o que for produzido
nessa matéria fique disponível para que mais pessoas possam ter acesso a esse
material, e que também possam contribuir no melhoramento do que estiver pronto,
bem como na geração de novas \emph{features}.

\textbf{Produzir uma \emph{gem}.} Seria uma forma de facilitar ainda mais a vida
do usuário programador. Outra vantagem seria a de propagar mais facilmente o
trabalho feito, atraindo,convidando mais mentes para a expansão desse projeto
futuramente.

\section{Metodologia}
\label{sec:methodology}

A metodologia seguida nesse trabalho consiste de alguns passos:
\begin{enumerate}
    \item Leitura de textos sobre o assunto e conversa com o orientador.
    \item Planejamento da estrutura do sistema.
    \item Implementação do sistema
    \item Documentação do processo e descrição de tudo o que tange o item anterior
    \item Se possível, criar uma documentação suficiente para a compreensão e
    utilização dessa ferramenta
\end{enumerate}

O item 1 foi o pontepé inicial dessa empreitada. Graças a meu orientador, grande
fã de ficção interativa, não tive grandes problemas para me interar a respeito
deste assunto. Meu maior desafio, entretanto, foi aprender uma linguagem para
que eu pudesse ter uma ideia do estado da arte e ter um parâmetro pra criar a
ferramenta. Entretanto, ao ler o livro \citet{Zil:95} indicado pelo professor,
percebi que ela era bem parecida com LISP, que já era conhecida da matéria
"Conceitos de linguagens de programação". Tal semelhaça foi de inestimável ajuda
para compreender aquela ferramenta.

O passo 2 consiste na esquematização dos arquivos a serem criados. É fundamental
para o planejamento do fluxo de trabalho a ser processado em tempo de execução.

Os passos 3 e 4 acontecem nessa fase final, e acontecem de forma quase paralela.
Tratam-se do coração da proposta feita meses antes. O passo 5 é decorrente de
todos os anteriores e ficará pronto depois de pronto o software.

\section{Descrição dos capítulos}
\label{sec:description}

O capítulo 2, \textbf{dos conceitos principais} funciona como um cabeçalho de um
programa. Isso pois, nessa parte do código é onde ficam os comandos de
importação de bibliotecas.

Analogamente, no próximo capítulo, tratar-se-ão de conceitos relativos ao
material a ser implementado, e do jargão a ser utilizado. Detalhes a serem
esclarecidos para que o cérebro do leitor "compile" (em outras palavras,
que ele compreenda) o conteúdo das páginas que se seguirão. Além disso, esse
capítulo traz uma breve história da Ficção Interativa, e uma apresentação
resumida das linguagens / ferramentas mais utilizadas (ZIL, Inform e TADS).

No capítulo 3, \textbf{da proposta}, tratar-se-ão de detalhes técnicos sobre a
implementação. Será uma descrição a dois níveis, sendo a primeira, uma descrição
verbal, e o seguinte, a nível de implementação.

No capítulo 4, \textbf{dos resultados obtidos}, encontra-se uma reflexão sobre o
que se passou além do código, o que está pronto e o que pode ser feito no futuro.

O último capítulo, \textbf{das conclusões finais}, faz-se um resumo do que está
documentado, destacando o que for mais conveniente.
