%% ------------------------------------------------------------------------- %%
\chapter{Introdução}
\label{cap:introducao}

Com o advento de tecnologias  ligadas à indústria de games, é possível perceber uma contante
preocupação com a imersão do jogador na história. Isso é ótimo, e o próprio autor se beneficia
com experiências mais realistas. Mas fica uma pergunta: será que jogos de mundo aberto, muitas
vezes capazes de transpor as barreiras temporais e espaciais poderão substituir os livros.
Pessoalmente, espero que não aconteça, mas é inegável o apelo desse jogos: a possiblidade do
jogador construir sua narrativa.

Os livros não oferecem essa possibilidade, mas oferecem um bem ainda maior: o hábito da leitura que
possibilita um ciclo virtuoso que envolve cultura e conhecimento e mais leitura. Seriam a diversão
e a leitura inconsiliáveis? Essa proposta de trabalho busca provar que não. O software a ser desenvolvido
permite que se criem aventuras de texto, também chamadas de ficção interativa que será explicada em mais
detalhes no capítulo sobre os conceitos principais.

Essa solução tem dois objetivos: criar uma ferramenta de geração de aventuras de texto de uma forma que
mesmo pessoas não familiarizadas com programação possam criar aventuras da forma como planejarem. Outro
objetivo importantissimo a considerar é criar um atrativo a mais para o hábito da leitura. Como cientista
da computação, não posso me arrogar a autoridade de dizer o que é ou não benéfico para crianças. Entretanto,
acredito que trazer a interatividade dos jogos para leitura pode ter um potencial para incentivar o hábito
da leitura.


\section{Objetivos Específicos}
\label{sec:goals}

Aqui tratarei de todos os objetivos de forma específica e como pretendo cumpri-los.

\textbf{Gerar divertimento}. O público desse software não é homogêneo. Tratam-se de pessoas criativas, mas não
exatamente familiarizados com a tarefa de gerar código; mas também de quem for entrar nessas aventuras.

Para o desenvolvedor, o atrativo será criar jogos de uma forma que deverá ser a mais intuitiva   

\section{Metodologia}
\label{sec:methodology}

\section{Descrição dos capítulos}
\label{sec:description}
%Uma monografia deve ter um cap�tulo inicial que � a Introdu��o e um
%cap�tulo final que � a Conclus�o. Entre esses dois cap�tulos poder�
%ter uma sequ�ncia de cap�tulos que descrevem o trabalho em detalhes.
%Ap�s o cap�tulo de conclus�o, poder� ter ap�ndices e ao final dever�
%ter as refer�ncias bibliogr�ficas.
%
%
%Para a escrita de textos em Ci�ncia da Computa��o, o livro de Justin Zobel,
%\emph{Writing for Computer Science} \citep{zobel:04} � uma leitura obrigat�ria.
%O livro \emph{Metodologia de Pesquisa para Ci�ncia da Computa��o} de
%\citet{waz:09} também merece uma boa lida.
%
%O uso desnecess�rio de termos em lingua estrangeira deve ser evitado. No entanto,
%quando isso for necess�rio, os termos devem aparecer \emph{em it�lico}.
%
%\begin{small}
%\begin{verbatim}
%Modos de cita��o:
%indesej�vel: [AF83] introduziu o algoritmo �timo.
%indesej�vel: (Andrew e Foster, 1983) introduziram o algoritmo �timo.
%certo : Andrew e Foster introduziram o algoritmo �timo [AF83].
%certo : Andrew e Foster introduziram o algoritmo �timo (Andrew e Foster, 1983).
%certo : Andrew e Foster (1983) introduziram o algoritmo �timo.
%\end{verbatim}
%\end{small}
%
%Uma pr�tica recomend�vel na escrita de textos � descrever as legendas das
%figuras e tabelas em forma auto-contida: as legendas devem ser razoavelmente
%completas, de modo que o leitor possa entender a figura sem ler o texto onde a
%figura ou tabela � citada.
%
%Apresentar os resultados de forma simples, clara e completa � uma tarefa que
%requer inspira��o. Nesse sentido, o livro de \citet{tufte01:visualDisplay},
%\emph{The Visual Display of Quantitative Information}, serve de ajuda na
%cria��o de figuras que permitam entender e interpretar dados/resultados de forma
%eficiente.
%
