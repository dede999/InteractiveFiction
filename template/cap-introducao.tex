%% ------------------------------------------------------------------------- %%
\chapter{Introdução}
\label{cap:introducao}

Com o advento de tecnologias  ligadas à indústria de games, é possível perceber uma contante
preocupação com a imersão do jogador na história. Isso é ótimo, e o próprio autor se beneficia
com experiências mais realistas. Mas fica uma pergunta: será que jogos de mundo aberto, muitas
vezes capazes de transpor as barreiras temporais e espaciais poderão substituir os livros.
Pessoalmente, espero que não aconteça, mas é inegável o apelo desse jogos: a possiblidade do
jogador construir sua narrativa.

Os livros não oferecem essa possibilidade, mas oferecem um bem ainda maior: o hábito da leitura que
possibilita um ciclo virtuoso que envolve cultura e conhecimento e mais leitura. Seriam a diversão
e a leitura inconsiliáveis? Essa proposta de trabalho busca provar que não. O software a ser desenvolvido
permite que se criem aventuras de texto, também chamadas de ficção interativa que será explicada em mais
detalhes no capítulo sobre os conceitos principais.

Essa solução tem dois objetivos: criar uma ferramenta de geração de aventuras de texto de uma forma que
mesmo pessoas não familiarizadas com programação possam criar aventuras da forma como planejarem. Outro
objetivo importantissimo a considerar é criar um atrativo a mais para o hábito da leitura. Como cientista
da computação, não posso me arrogar a autoridade de dizer o que é ou não benéfico para crianças. Entretanto,
acredito que trazer a interatividade dos jogos para leitura pode ter um potencial para incentivar o hábito
da leitura.

% Para mais detalhes, fica disponível neste \href{linux.ime.usp.br/\~andreluizas}{\textbf{link}} mais


\section{Objetivos Específicos}
\label{sec:goals}

Aqui tratarei de todos os objetivos de forma específica e como pretendo cumpri-los.

\textbf{Gerar divertimento para o público} desse software. Este público não é homogêneo, e engloba
pessoas criativas, mas não exatamente familiarizados com a tarefa de gerar código. Além desse grupo,
há também aqueles que irão se divertir com as aventuras criadas.

Para o desenvolvedor, o atrativo será criar jogos de uma forma que deverá ser a mais intuitiva possível.
Para tal, inspirei-me no estilo de programar do \emph{RSpec}, e também na forma de estruturar os elementos
do \emph{ZIL}. A ideia é trabalhar com blocos descritivos com tokens de começo e fim (que torne esses
elementos o mais explícitos possível) e no meio destes somente pares de chaves e seus respectivos valores.

Para o jogador é crucial resguardar a experiência de jogo de clássicos como as que se obtem na série de
aventuras de texto \emph{Zork} \citep{Zork}. Uma ideia lançada pelo orientador permitira que se inserissem
sons e planos de fundo ou imagens. É uma boa adição ao sistema, mas não é a prioridade.

\textbf{Compartilhar conhecimento.} É fundamental que tudo o que for produzido nessa matéria fique disponível
para que mais pessoas possam ter acesso a esse material, e que também possam contribuir no melhoramento do
que estiver pronto, bem como na geração de novas \emph{features}.

\textbf{Produzir uma \emph{gem}.} Seria uma forma de facilitar ainda mais a vida do usuário, ainda que este tipo
específico esteja mais a vontade

\section{Metodologia}
\label{sec:methodology}

A metodologia seguida nesse trabalho consiste de alguns passos: \begin{enumerate}
  \item Leitura de textos sobre o assunto e conversa com o orientador.
  \item Planejamento da estrutura do sistema.
  \item Implementação do sistema
  \item Documentação do processo e descrição de tudo o que tange o item anterior
  \item Se possível, criar uma documentação suficiente para a compreensão e utilização dessa ferramenta
\end{enumerate}

O item 1 foi o pontepé inicial dessa empreitada. Graças a meu orientador, grande fã de ficção interativa, não tive
grandes problemas para me interar a respeito desse assunto. Meu maior desafio foi aprender uma linguagem para que
eu pudesse ter uma ideia do estado da arte e ter um parâmetro pra criar as ferramentas. Entretanto, ao ler o livro
\citet{Zil:95} indicado pelo professor, percebi que ela era bem parecida com LISP, que já era conhecida da matéria
"Conceitos de linguagens de programação".

O passo 2 consiste na esquematização dos arquivos a serem criados. É fundamentalpara o planejamento do fluxo de
trabalho a ser processado em tempo de execução.

Os passos 3 e 4 acontecem nessa fase final, e acontecem de forma quase paralela. Tratam-se do coração da proposta feita
meses antes. O passo 5 é decorrente de todos os anteriores e ficará pronto depois de pronto o software.

\section{Descrição dos capítulos}
\label{sec:description}

O capítulo 2, \textbf{dos conceitos principais} funciona como um cabeçalho de um programa. Tal comparação ocorre,
pois nessa parte do código é onde ficam os comandos de importação de bibliotecas. Analogamente, no próximo capítulo,
tratar-se-ão de conceitos relativos ao material a ser implementado, e do jargão a ser utilizado. Detalhes a serem
esclarecidos para que o cérebro do leitor "compile", ou seja, compreenda, o conteúdo das páginas que se seguirão.

No capítulo 3, \textbf{do estado da arte}, tratarei brevemente de outras formas de desenvolver Aventuras de Texto,
comparando-as entre si e o que pretende-se criar neste projeto.

No capítulo 4, \textbf{da proposta}, tratar-se-á de detalhes técnicos sobre a implementação. Soluções, partes do
problema entre outros fatores cruciais pra que seja compreensível para o leitor como funciona o sistema de forma geral.

No capítulo 5, \textbf{dos resultados obtidos}, encontra-se
