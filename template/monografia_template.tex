% Arquivo LaTeX de exemplo de monografia para a disciplina MAC0499
%
% Adaptado em julho/2015 a partir do
%
% ---------------------------------------------------------------------------- %
% Arquivo LaTeX de exemplo de dissertação/tese a ser apresentados à CPG do IME-USP
%
% Versão 5: Sex Mar  9 18:05:40 BRT 2012
%
% Criação: Jesús P. Mena-Chalco
% Revisão: Fabio Kon e Paulo Feofiloff


\documentclass[12pt,twoside,a4paper]{book}


% ---------------------------------------------------------------------------- %
% Pacotes
\usepackage[T1]{fontenc}
\usepackage[brazil]{babel}
\usepackage[utf8]{inputenc}
\usepackage[pdftex]{graphicx}           % usamos arquivos pdf/png como figuras
\usepackage{setspace}                   % espaçamento flexível
\usepackage{indentfirst}                % indentação do primeiro parágrafo
\usepackage{makeidx}                    % índice remissivo
\usepackage[nottoc]{tocbibind}          % acrescentamos a bibliografia/indice/conteudo no Table of Contents
\usepackage{courier}                    % usa o Adobe Courier no lugar de Computer Modern Typewriter
\usepackage{type1cm}                    % fontes realmente escaláveis
\usepackage{listings}                   % para formatar código-fonte (ex. em Java)
\usepackage{titletoc}
%\usepackage[bf,small,compact]{titlesec} % cabeçalhos dos títulos: menores e compactos
\usepackage[fixlanguage]{babelbib}
\usepackage[font=small,format=plain,labelfont=bf,up,textfont=it,up]{caption}
\usepackage[usenames,svgnames,dvipsnames]{xcolor}
\usepackage[a4paper,top=2.54cm,bottom=2.0cm,left=2.0cm,right=2.54cm]{geometry} % margens
%\usepackage[pdftex,plainpages=false,pdfpagelabels,pagebackref,colorlinks=true,citecolor=black,linkcolor=black,urlcolor=black,filecolor=black,bookmarksopen=true]{hyperref} % links em preto
\usepackage[pdftex,plainpages=false,pdfpagelabels,pagebackref,colorlinks=true,citecolor=DarkGreen,linkcolor=NavyBlue,urlcolor=DarkRed,filecolor=green,bookmarksopen=true]{hyperref} % links coloridos
\usepackage[all]{hypcap}                    % soluciona o problema com o hyperref e capitulos
\usepackage[round,sort,nonamebreak]{natbib} % citação bibliográfica textual(plainnat-ime.bst)
\usepackage{emptypage}  % para não colocar número de página em página vazia
\fontsize{60}{62}\usefont{OT1}{cmr}{m}{n}{\selectfont}

% ---------------------------------------------------------------------------- %
% Cabeçalhos similares ao TAOCP de Donald E. Knuth
\usepackage{fancyhdr}
\pagestyle{fancy}
\fancyhf{}
\renewcommand{\chaptermark}[1]{\markboth{\MakeUppercase{#1}}{}}
\renewcommand{\sectionmark}[1]{\markright{\MakeUppercase{#1}}{}}
\renewcommand{\headrulewidth}{0pt}

% ---------------------------------------------------------------------------- %
\graphicspath{{./figuras/}}             % caminho das figuras (recomendável)
\frenchspacing                          % arruma o espaço: id est (i.e.) e exempli gratia (e.g.)
\urlstyle{same}                         % URL com o mesmo estilo do texto e não mono-spaced
\makeindex                              % para o índice remissivo
\raggedbottom                           % para não permitir espaços extra no texto
\fontsize{60}{62}\usefont{OT1}{cmr}{m}{n}{\selectfont}
\cleardoublepage
\normalsize

% ---------------------------------------------------------------------------- %
% Opções de listing usados para o código fonte
% Ref: http://en.wikibooks.org/wiki/LaTeX/Packages/Listings
\lstset{ %
language=Ruby,                  % choose the language of the code
basicstyle=\footnotesize,       % the size of the fonts that are used for the code
numbers=left,                   % where to put the line-numbers
numberstyle=\footnotesize,      % the size of the fonts that are used for the line-numbers
stepnumber=1,                   % the step between two line-numbers. If it's 1 each line will be numbered
numbersep=5pt,                  % how far the line-numbers are from the code
showspaces=false,               % show spaces adding particular underscores
showstringspaces=false,         % underline spaces within strings
showtabs=false,                 % show tabs within strings adding particular underscores
frame=single,	                % adds a frame around the code
framerule=0.6pt,
tabsize=2,	                    % sets default tabsize to 2 spaces
captionpos=b,                   % sets the caption-position to bottom
breaklines=true,                % sets automatic line breaking
breakatwhitespace=false,        % sets if automatic breaks should only happen at whitespace
escapeinside={\%*}{*)},         % if you want to add a comment within your code
backgroundcolor=\color[rgb]{1.0,1.0,1.0}, % choose the background color.
rulecolor=\color[rgb]{0.8,0.8,0.8},
extendedchars=true,
xleftmargin=10pt,
xrightmargin=10pt,
framexleftmargin=10pt,
framexrightmargin=10pt
}
\newcommand{\codigo}[5]{\lstinputlisting[firstline={#1},lastline={#2}, caption={#3}, label={#4}]{#5}}

% ---------------------------------------------------------------------------- %
% Corpo do texto
\begin{document}

    \frontmatter
    % cabeçalho para as páginas das seções anteriores ao capítulo 1 (frontmatter)
    \fancyhead[RO]{{\footnotesize\rightmark}\hspace{2em}\thepage}
    \setcounter{tocdepth}{2}
    \fancyhead[LE]{\thepage\hspace{2em}\footnotesize{\leftmark}}
    \fancyhead[RE,LO]{}
    \fancyhead[RO]{{\footnotesize\rightmark}\hspace{2em}\thepage}

    \onehalfspacing  % espaçamento


% ---------------------------------------------------------------------------- %
% CAPA
    \thispagestyle{empty}
    \begin{center}
        \vspace*{2.3cm}
        Universidade de São Paulo\\
        Instituto de Matemática e Estatística \\
        Bacharelado  em Ciência da Computação


        \vspace*{3cm}
        \Large{André Luiz Abdalla Silveira}


        \vspace{3cm}
        \textbf{\Large{Ficção Interativa \\
            levando diversão e cultura além do código}}


        \vskip 5cm
        \normalsize{São Paulo}

        \normalsize{Novembro de 2018}
    \end{center}
    % ---------------------------------------------------------------------------- %
    % Página de rosto
    %
    \newpage
    \thispagestyle{empty}
    \begin{center}
        \vspace*{2.3 cm}
        \textbf{\Large{Ficção Interativa \\
            levando diversão e cultura além do código}}
        \vspace*{2 cm}
    \end{center}

    \vskip 2cm

    \begin{flushright}
        Monografia final da disciplina \\
        MAC0499 -- Trabalho de Formatura Supervisionado.
    \end{flushright}

    \vskip 5cm

    \begin{center}
        Supervisor: Prof. Dr. Marco Dimas Gubitoso

        \vskip 5cm
        \normalsize{São Paulo}

        \normalsize{Novembro de 2018}
    \end{center}
    \pagebreak




    \pagenumbering{roman}     % começamos a numerar

    % ---------------------------------------------------------------------------- %
    % Agradecimentos:
    \chapter*{Agradecimentos}

    Meus agradecimentos pricipais são a Deus, meus adorados pais que me proporcianram
    as melhores oportunidades que eu poderia ter. Meu agradecimento também dirige-se
    ao meu orientador que me apoiou nos momentos inicais do meu trabalho, e também me
    ajudou bastante em momentos em que eu não sabia exatamente o que fazer nessa
    matéria, e que me deu essa ideia de trabalho.



    % ---------------------------------------------------------------------------- %
    % Resumo
    \chapter*{Resumo}


    Este trabalho se trata da implementação de um software que, a partir de uma entrada em linguagem
    natural, (em Português, mas podendo ser expandida em futuras iterações) e gera um jogo onde
    tudo que é necessário é leitura, imaginação e um computador.

    Tal programa será desenvolvido em Ruby \footnote{assim como seus testes que usam RSpec}, dado a
    praticidade ao lidar com expressões rugulares e interpretadores de texto, indispensáveis para
    essa 'aventura'
    \\


    \noindent \textbf{Palavras-chave:} ruby, ficcao-interetiva, rspec

    % ---------------------------------------------------------------------------- %
    % Abstract
    \chapter*{Abstract}


    This paper is about a software implementation. Its job is processing natural language, (in Portuguese
    at first, but it can be expanded on future itarations) and return a game where everything you need is
    reading, imagination and your computer.

    This program will be written in Ruby \footnote{such as its tests that use RSpec}, because of the practicality
    on handling with regular expressions and text interpreters, which are crucial for this 'adventure'
    \\

    \noindent \textbf{Keywords:} ruby, interactive-ficction, rspec.



    % ---------------------------------------------------------------------------- %
    % Sumário
    \tableofcontents    % imprime o sumário




    %% % ---------------------------------------------------------------------------- %
    %% \chapter{Lista de Abreviaturas}
    %% \begin{tabular}{ll}
    %%          CFT         & Transformada contínua de Fourier (\emph{Continuous Fourier Transform})\\
    %%          DFT         & Transformada discreta de Fourier (\emph{Discrete Fourier Transform})\\
    %%         EIIP         & Potencial de interação elétron-íon (\emph{Electron-Ion Interaction Potentials})\\
    %%         STFT         & Tranformada de Fourier de tempo reduzido (\emph{Short-Time Fourier Transform})\\
    %% \end{tabular}

    %% % ---------------------------------------------------------------------------- %
    %% \chapter{Lista de Símbolos}
    %% \begin{tabular}{ll}
    %%         $\omega$    & Frequência angular\\
    %%         $\psi$      & Função de análise \emph{wavelet}\\
    %%         $\Psi$      & Transformada de Fourier de $\psi$\\
    %% \end{tabular}

    %% % ---------------------------------------------------------------------------- %
    %% % Listas de figuras e tabelas criadas automaticamente
    %% \listoffigures
    %% \listoftables



    % ---------------------------------------------------------------------------- %
    % Capítulos do trabalho
    \mainmatter

    % cabeçalho para as páginas de todos os capítulos
    \fancyhead[RE,LO]{\thesection}

    \singlespacing              % espaçamento simples
    %\onehalfspacing            % espaçamento um e meio


    \input cap-introducao
    \input cap-conceitos
    \input cap-estado_arte
    \input cap-proposta
    \input cap-resultados
    \input cap-conclusoes

    % cabeçalho para os apêndices
    \renewcommand{\chaptermark}[1]{\markboth{\MakeUppercase{\appendixname\ \thechapter}} {\MakeUppercase{#1}} }
    \fancyhead[RE,LO]{}
    \appendix

    \chapter{T�tulo do ap�ndice}
\label{cap:ape}

Texto texto texto texto texto texto texto texto texto texto texto texto texto
texto texto texto texto texto texto texto texto texto texto texto texto texto
texto texto texto texto texto texto.

      % associado ao arquivo: 'ape-conjuntos.tex'


    % ---------------------------------------------------------------------------- %
    % Bibliografia
    \backmatter \singlespacing   % espaçamento simples
    \bibliographystyle{plainnat-ime} % citação bibliográfica textual
    \bibliography{bibliografia}  % associado ao arquivo: 'bibliografia.bib'


    %%%  ---------------------------------------------------------------------------- %
    %% % Índice remissivo
    %% \index{TBP|see{periodicidade região codificante}}
    %% \index{DSP|see{processamento digital de sinais}}
    %% \index{STFT|see{transformada de Fourier de tempo reduzido}}
    %% \index{DFT|see{transformada discreta de Fourier}}
    %% \index{Fourier!transformada|see{transformada de Fourier}}

    %% \printindex   % imprime o índice remissivo no documento

\end{document}
