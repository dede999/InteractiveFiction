\chapter{Da Proposta}
\label{cap:propose}

\section{Qual é a proposta?}
\label{sec:what_is}

A proposta desse trabalho não é complexa. Trata-se da criação de linguagem de
domínio específico (DSL em inglês). Com essa linguagem estruturada, nossa meta é
desenvolver o compilador que interpretará seus comandos, alocando as estruturas
de dados necesárias para a execução do programa. Esses dados ficam na máquina
até o jogo acabar. Depois disso, os dados são descartados ficando apenas o
script do jogo. É o que se propõe para a entrega final \footnote{vide figura
\ref{fig:project}}. Isso não significa, entretanto, que o trabalho deva ficar
parado aí. Pelo contrário, deseja-se criar uma cadeia de processamento que
acredito ser melhor do que a atual como explicado na sessão
\ref{sec:what-can-be-done}

\begin{figure}[htb]
  \centering
  \includegraphics[width=8.5cm]{figuras/project}
  \caption{\label{fig:project} Como funciona o programa.}
\end{figure}


\section{Como funciona?}
\label{sec:how-it-works}

Nessa parte do capítulo, fica a explicação de como funciona o software global e
localmente. Primeiro, apresentam-se as gramáticas (para a criação e depois, para
a execução). Depois, os arquivos serão explicados dentro de suas funções. Por
último, fica a descrição de como se encaixam para o funcionamento geral.

\subsection{As gramáticas}
\label{subsec:grammar}

\subsubsection{Para o parser}
\label{subsubsec:parser}

A gramática a ser seguida no interpretador do jogo é muito similar ao exemplo da
figura \ref{fig:rspec}. Entretanto, considerando que esta apresenta outras
influências e propriedades, é fundamental que seja apresentada.

\paragrafo{Cabeçalho}: Parte capital da fonte do jogo, resposável por definir os
metadados do jogo, como nome, autor(es), versão, e outros dados que se acreditar
relevantes. Esses meta dados também podem ser encarados como variáveis globais,
a serem mencionadas a frente, mas a diferença capital é que meta dados serão
exibidos no início do jogo, antes mesmo da apresentação do menu.

\vspace{1cm}

\begin{tabular}{|l|}
  \hline
  \texttt{\kw{meta} autor = Josefino} \\
  \texttt{\kw{meta} title = O jogo do Ano} \\
  \texttt{\kw{meta} desc = O melhor jogo que você vai jogar nesse ano} \\
  \quad \quad \quad \quad \texttt{ou talvez nos próximos 10 min} \\
  \hline
\end{tabular}

\vspace{1cm}

\paragrafo{Variáveis}: Definições de propriedades básicas do jogo (como a
quantidade de pontos de vida, força, número do turno) e outras que o jogador vai
criar. As variáveis podem ser de três tipos: \begin{description}
  \item[\kw{meta}] -- Meta dados -- como mencionado acima, podem ser considerados
  como variáveis globais, porque podem ser utilizadas como parâmetro e em
  qualquer escopo. Entretanto, diferenciam-se das variáveis globais descritas
  abaixo pois não podem ter seu valor alterado enquanto o jogo está sendo
  executado
  \item[\kw{var}] -- Variáveis globais -- Podem ser usadas em qualquer escopo, e
  podem ter seu valor alterado em tempo de execução.
  \item[\kw{let}] -- Variáveis de escopo (funções) --  válidas somente no escopo em que
  estão definidas. Nada as impediria de ser usada em um escopo global, mas seu
  uso é mais adequado dentro de funções e laços específicos. \footnote{perguntar
  ao professor se é conveniente ou proveitoso fazer essa distinção.}
\end{description}

\paragrafo{Objetos}


\paragrafo{Lugares}


\paragrafo{Funções}


\paragrafo{Eventos}


\paragrafo{NPCs e demais personagens}


\subsubsection{Para o jogo}
\label{subsubsec:game}

\subsection{Arquivos}
\label{subsec:files}

\begin{itemize}
  \item \texttt{\textbf{game.rb}}: esse arquivo é responsável pela
  metaprogramação. Ou seja, por programar a programação. Quer dizer que a sua
  programação é a responsável por criar eventos, alocar dados e suas estruturas
  de forma que o jogo se processe da melhor forma. Contém classes referentes aos
  personagens, eventos, objetos e ambientes.
  \item \texttt{\textbf{run.rb}}: responsável por executar o jogo. Cabe decisão
  se este arquivo ficará em um módulo (mais provável) ou em uma classe. Mas é
  certo que a programação nele contida será dirigida para fazer o jogo funcionar,
  isto é, passar turnos, alterar inventário, mover o jogador e objetos e demais
  eventos.
  \item \texttt{\textbf{parser.rb}}: responsável por interpretar o arquivo de
  jogo. Aqui entra em ação uma das melhores particularidades do Ruby
  \footnote{no contexto desse trabalho, como citado na sessão \ref{sec:ruby}}: a
  sua praticidade em trabalhar com texto. Esse arquivo interpretará as entradas
  do jogador também.
  \item \texttt{\textbf{start.rb}}: como diz o nome, é responsável por começar a
  execução. Aciona os demais módulos e classes.
\end{itemize}

\subsection{Como trabalham}
\label{subsec:how-they-work}

O funcionamento do sistema acontece num fluxo linear e muito simples. Começa com
a leitura do arquivo $\mathtt{*.fi}$\footnote{ficção interativa}, e acaba quando
o jogador decide não jogar mais. Essa sessão existe para contar o que acontece,
e como os arquivos trabalham entre si.

O primeiro passo é o desenvolvedor gerando o arquivo com a descrição do seu jogo
usando as regras a serem apresentadas no capítulo de resultados. Ainda que fosse
muito desejável fazer uma linguagem mais próxima da natural, tal como
\emph{Inform7}, é totalmente inviável dado não somente o tempo para conclusão
(cerca de 1 ano), mas como o fato de que seria exigido um estudo muito mais
profundo na área de desenvolvimento em linguagem natural. Em outras palavras, é
desejável, mas não é viável.

Depois disso, aciona-se \texttt{\textbf{start.rb}}. Ele funciona tal qual uma
ignição ao acionar as demais partes do programa. A primeira dessas partes é o
interpretador, o \texttt{\textbf{parser.rb}}. Esse módulo, num primeiro momento,
compila o que o criador do jogo determina para o seu jogo. Conforme o arquivo
base é lido, o parser procura reconhecer o elemento que o usuário vai incluir,
costurando-os um a um, de maneira a dar vida à aventura planejada. Depois de
compilar o jogo, o interpretador toma a responsabilidade de interpretar o que o
jogador escreve, retornando para cada linha de comando, uma resposta ou uma ação.

Isso é possível por causa do arquivo \texttt{\textbf{game.rb}}. Toda vez que o
autor declarar que quer construir um elemento, o prgrama cria uma instância da
estrutura respectiva. Conforme lê os atributos, o programa atribui a essa
instância o valor determinado, se e somente se, a estrutura tiver esse atributo.
Do contrário, uma mensagem de aviso é lançada, e aquela atribuição estranha
é ignorada.

Um ponto importante a levar em conta são o que posso chamar de "referências
futuras". Isto é, nada impede o autor de dizer que a cozinha fica do lado da
sala, ainda que um desses cômodos ainda não tenham sido declarado. Nesses casos,
referências futuras serão armazenadas e deixadas para o momento depois da
leitura. Momento no qual, hipoteticamente, todos os elementos do jogo estarão
declarados. Entretanto, se ainda houver ambientes sem comunicação com o resto do
jogo, ou elementos soltos, uma mensagem de aviso poderá sugir, mas estuda-se
também perguntar ao autor do jogo se não deseja completar esses detealhes que
ficaram pra trás.

Terminada a compilação, o arquivo \texttt{\textbf{run.rb}} começa o seu trabalho.
O jogo começa perguntando ao jogador se quer começar a aventura. Caso afirmativo,
a história se inicia. Esse menu de jogo também apresenta opção de saída também,
bem como de conhecer a sinopse do jogo e opções como a de criar um arquivo
\footnote{com o desenvolvimento da narrativa} no computador ou não. Tal arquivo,
além de servir como um troféu por ter vencido na história (quando acontecer),
visa gerar no jogador a sensação de ter parte naquela história. Em última
instância, que fique registrado o quanto o autor desse trabalho dejesa estimular
nas pessoas o ímpeto de criar novas histórias, seja a partir dos seus passos no
jogo, seja em \emph{fan-fics}\footnote{a grosso modo, trata-se de histórias
criadas a partir de outras, por fãs  destas. Fica como sugestão de leitura,
por lazer, as histórias criadas por fãs da célebre série da autora J. K. Rowling
nesse
\href{https://super.abril.com.br/blog/turma-do-fundao/5-fanfics-para-matar-a-saudade-de-harry-potter/}{\textbf{link}}},
ou suas pŕoprias narrativas.

No capítulo de resultados, encontra-se\footnote{depois do trabalho acabado} um
tutorial de como usar a ferramenta, do que pode ser feito com ela, e o que mais
poderá ser feito.
