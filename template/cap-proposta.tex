\chapter{Da Proposta}
\label{cap:propose}

\section{Qual é a proposta?}
\label{sec:what_is}

A proposta desse trabalho não é complexa. Trata-se da criação de linguagem de domínio específico (DSL em inglês).
Com essa linguagem estruturada, nossa meta é desenvolver o compilador que interpretará seus comandos, alocando as estruturas
de os dados necesários para a execução. Esses dados ficam na máquina até o jogo acabar. Depois disso, os dados são descartados ficando
apenas o script do jogo. É o que se propõe para a entrega final \footnote{vide figura \ref{fig:project}}. Isso não significa, entretanto, que o trabalho deva ficar parado aí. Pelo contrário, deseja-se criar uma cadeia de processamento que acredito ser melhor como explicado na sessão \ref{sec:what-can-be-done}

\begin{figure}[htb]
  \centering
  \includegraphics[width=8.5cm]{figuras/project}
  \caption{\label{fig:project} Como funciona o programa.}
\end{figure}


\section{Como funciona?}
\label{sec:how-it-works}

Nessa parte do capítulo, fica a explicação de como funciona o software global e localmente. Primeiro, os arquivos serão explicados dentro de suas funções. Por último, fica a descrição de como se encaixam para o funcionamento geral.

\subsection{Arquivos}
\label{subsec:files}

\begin{itemize}
  \item \texttt{\textbf{game.rb}}\footnote{único a existir na data da entrega}: esse arquivo é responsável pela metaprogramação. Ou seja, por programar os eventos, alocar dados e suas estruturas de forma que o jogo se processe da melhor forma. Contém classes referentes aos personagens, eventos, objetos e ambientes.
  \item \texttt{\textbf{run.rb}}: responsável por executar o jogo. Cabe decisão se este arquivo abrigará um módulo (mais provável) ou uma classe. Mas é certo que a programação nele contida será dirigida para fazer o jogo funcionar, isto é, passar turnos, alterar inventário, mover o jogador e objetos e demais eventos.
  \item \texttt{\textbf{parser.rb}}: responsável por interpretar o arquivo de jogo. Aqui entra em ação uma das melhores particularidades do Ruby\footnote{no contexto desse trabalho, como citado na sessão \ref{sec:ruby}}: a sua praticidade em trabalhar com texto. Esse arquivo interpretará as entradas do jogador também.
\end{itemize}

\subsection{Como trabalham}
\label{subsec:how-they-work}
